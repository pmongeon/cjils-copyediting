% Options for packages loaded elsewhere
\PassOptionsToPackage{unicode}{hyperref}
\PassOptionsToPackage{hyphens}{url}
\PassOptionsToPackage{dvipsnames,svgnames,x11names}{xcolor}
%
\documentclass[
  letterpaper,
  DIV=11,
  numbers=noendperiod]{scrreprt}

\usepackage{amsmath,amssymb}
\usepackage{iftex}
\ifPDFTeX
  \usepackage[T1]{fontenc}
  \usepackage[utf8]{inputenc}
  \usepackage{textcomp} % provide euro and other symbols
\else % if luatex or xetex
  \usepackage{unicode-math}
  \defaultfontfeatures{Scale=MatchLowercase}
  \defaultfontfeatures[\rmfamily]{Ligatures=TeX,Scale=1}
\fi
\usepackage{lmodern}
\ifPDFTeX\else  
    % xetex/luatex font selection
\fi
% Use upquote if available, for straight quotes in verbatim environments
\IfFileExists{upquote.sty}{\usepackage{upquote}}{}
\IfFileExists{microtype.sty}{% use microtype if available
  \usepackage[]{microtype}
  \UseMicrotypeSet[protrusion]{basicmath} % disable protrusion for tt fonts
}{}
\makeatletter
\@ifundefined{KOMAClassName}{% if non-KOMA class
  \IfFileExists{parskip.sty}{%
    \usepackage{parskip}
  }{% else
    \setlength{\parindent}{0pt}
    \setlength{\parskip}{6pt plus 2pt minus 1pt}}
}{% if KOMA class
  \KOMAoptions{parskip=half}}
\makeatother
\usepackage{xcolor}
\setlength{\emergencystretch}{3em} % prevent overfull lines
\setcounter{secnumdepth}{5}
% Make \paragraph and \subparagraph free-standing
\makeatletter
\ifx\paragraph\undefined\else
  \let\oldparagraph\paragraph
  \renewcommand{\paragraph}{
    \@ifstar
      \xxxParagraphStar
      \xxxParagraphNoStar
  }
  \newcommand{\xxxParagraphStar}[1]{\oldparagraph*{#1}\mbox{}}
  \newcommand{\xxxParagraphNoStar}[1]{\oldparagraph{#1}\mbox{}}
\fi
\ifx\subparagraph\undefined\else
  \let\oldsubparagraph\subparagraph
  \renewcommand{\subparagraph}{
    \@ifstar
      \xxxSubParagraphStar
      \xxxSubParagraphNoStar
  }
  \newcommand{\xxxSubParagraphStar}[1]{\oldsubparagraph*{#1}\mbox{}}
  \newcommand{\xxxSubParagraphNoStar}[1]{\oldsubparagraph{#1}\mbox{}}
\fi
\makeatother

\usepackage{color}
\usepackage{fancyvrb}
\newcommand{\VerbBar}{|}
\newcommand{\VERB}{\Verb[commandchars=\\\{\}]}
\DefineVerbatimEnvironment{Highlighting}{Verbatim}{commandchars=\\\{\}}
% Add ',fontsize=\small' for more characters per line
\usepackage{framed}
\definecolor{shadecolor}{RGB}{241,243,245}
\newenvironment{Shaded}{\begin{snugshade}}{\end{snugshade}}
\newcommand{\AlertTok}[1]{\textcolor[rgb]{0.68,0.00,0.00}{#1}}
\newcommand{\AnnotationTok}[1]{\textcolor[rgb]{0.37,0.37,0.37}{#1}}
\newcommand{\AttributeTok}[1]{\textcolor[rgb]{0.40,0.45,0.13}{#1}}
\newcommand{\BaseNTok}[1]{\textcolor[rgb]{0.68,0.00,0.00}{#1}}
\newcommand{\BuiltInTok}[1]{\textcolor[rgb]{0.00,0.23,0.31}{#1}}
\newcommand{\CharTok}[1]{\textcolor[rgb]{0.13,0.47,0.30}{#1}}
\newcommand{\CommentTok}[1]{\textcolor[rgb]{0.37,0.37,0.37}{#1}}
\newcommand{\CommentVarTok}[1]{\textcolor[rgb]{0.37,0.37,0.37}{\textit{#1}}}
\newcommand{\ConstantTok}[1]{\textcolor[rgb]{0.56,0.35,0.01}{#1}}
\newcommand{\ControlFlowTok}[1]{\textcolor[rgb]{0.00,0.23,0.31}{\textbf{#1}}}
\newcommand{\DataTypeTok}[1]{\textcolor[rgb]{0.68,0.00,0.00}{#1}}
\newcommand{\DecValTok}[1]{\textcolor[rgb]{0.68,0.00,0.00}{#1}}
\newcommand{\DocumentationTok}[1]{\textcolor[rgb]{0.37,0.37,0.37}{\textit{#1}}}
\newcommand{\ErrorTok}[1]{\textcolor[rgb]{0.68,0.00,0.00}{#1}}
\newcommand{\ExtensionTok}[1]{\textcolor[rgb]{0.00,0.23,0.31}{#1}}
\newcommand{\FloatTok}[1]{\textcolor[rgb]{0.68,0.00,0.00}{#1}}
\newcommand{\FunctionTok}[1]{\textcolor[rgb]{0.28,0.35,0.67}{#1}}
\newcommand{\ImportTok}[1]{\textcolor[rgb]{0.00,0.46,0.62}{#1}}
\newcommand{\InformationTok}[1]{\textcolor[rgb]{0.37,0.37,0.37}{#1}}
\newcommand{\KeywordTok}[1]{\textcolor[rgb]{0.00,0.23,0.31}{\textbf{#1}}}
\newcommand{\NormalTok}[1]{\textcolor[rgb]{0.00,0.23,0.31}{#1}}
\newcommand{\OperatorTok}[1]{\textcolor[rgb]{0.37,0.37,0.37}{#1}}
\newcommand{\OtherTok}[1]{\textcolor[rgb]{0.00,0.23,0.31}{#1}}
\newcommand{\PreprocessorTok}[1]{\textcolor[rgb]{0.68,0.00,0.00}{#1}}
\newcommand{\RegionMarkerTok}[1]{\textcolor[rgb]{0.00,0.23,0.31}{#1}}
\newcommand{\SpecialCharTok}[1]{\textcolor[rgb]{0.37,0.37,0.37}{#1}}
\newcommand{\SpecialStringTok}[1]{\textcolor[rgb]{0.13,0.47,0.30}{#1}}
\newcommand{\StringTok}[1]{\textcolor[rgb]{0.13,0.47,0.30}{#1}}
\newcommand{\VariableTok}[1]{\textcolor[rgb]{0.07,0.07,0.07}{#1}}
\newcommand{\VerbatimStringTok}[1]{\textcolor[rgb]{0.13,0.47,0.30}{#1}}
\newcommand{\WarningTok}[1]{\textcolor[rgb]{0.37,0.37,0.37}{\textit{#1}}}

\providecommand{\tightlist}{%
  \setlength{\itemsep}{0pt}\setlength{\parskip}{0pt}}\usepackage{longtable,booktabs,array}
\usepackage{calc} % for calculating minipage widths
% Correct order of tables after \paragraph or \subparagraph
\usepackage{etoolbox}
\makeatletter
\patchcmd\longtable{\par}{\if@noskipsec\mbox{}\fi\par}{}{}
\makeatother
% Allow footnotes in longtable head/foot
\IfFileExists{footnotehyper.sty}{\usepackage{footnotehyper}}{\usepackage{footnote}}
\makesavenoteenv{longtable}
\usepackage{graphicx}
\makeatletter
\newsavebox\pandoc@box
\newcommand*\pandocbounded[1]{% scales image to fit in text height/width
  \sbox\pandoc@box{#1}%
  \Gscale@div\@tempa{\textheight}{\dimexpr\ht\pandoc@box+\dp\pandoc@box\relax}%
  \Gscale@div\@tempb{\linewidth}{\wd\pandoc@box}%
  \ifdim\@tempb\p@<\@tempa\p@\let\@tempa\@tempb\fi% select the smaller of both
  \ifdim\@tempa\p@<\p@\scalebox{\@tempa}{\usebox\pandoc@box}%
  \else\usebox{\pandoc@box}%
  \fi%
}
% Set default figure placement to htbp
\def\fps@figure{htbp}
\makeatother
% definitions for citeproc citations
\NewDocumentCommand\citeproctext{}{}
\NewDocumentCommand\citeproc{mm}{%
  \begingroup\def\citeproctext{#2}\cite{#1}\endgroup}
\makeatletter
 % allow citations to break across lines
 \let\@cite@ofmt\@firstofone
 % avoid brackets around text for \cite:
 \def\@biblabel#1{}
 \def\@cite#1#2{{#1\if@tempswa , #2\fi}}
\makeatother
\newlength{\cslhangindent}
\setlength{\cslhangindent}{1.5em}
\newlength{\csllabelwidth}
\setlength{\csllabelwidth}{3em}
\newenvironment{CSLReferences}[2] % #1 hanging-indent, #2 entry-spacing
 {\begin{list}{}{%
  \setlength{\itemindent}{0pt}
  \setlength{\leftmargin}{0pt}
  \setlength{\parsep}{0pt}
  % turn on hanging indent if param 1 is 1
  \ifodd #1
   \setlength{\leftmargin}{\cslhangindent}
   \setlength{\itemindent}{-1\cslhangindent}
  \fi
  % set entry spacing
  \setlength{\itemsep}{#2\baselineskip}}}
 {\end{list}}
\usepackage{calc}
\newcommand{\CSLBlock}[1]{\hfill\break\parbox[t]{\linewidth}{\strut\ignorespaces#1\strut}}
\newcommand{\CSLLeftMargin}[1]{\parbox[t]{\csllabelwidth}{\strut#1\strut}}
\newcommand{\CSLRightInline}[1]{\parbox[t]{\linewidth - \csllabelwidth}{\strut#1\strut}}
\newcommand{\CSLIndent}[1]{\hspace{\cslhangindent}#1}

\KOMAoption{captions}{tableheading}
\makeatletter
\@ifpackageloaded{tcolorbox}{}{\usepackage[skins,breakable]{tcolorbox}}
\@ifpackageloaded{fontawesome5}{}{\usepackage{fontawesome5}}
\definecolor{quarto-callout-color}{HTML}{909090}
\definecolor{quarto-callout-note-color}{HTML}{0758E5}
\definecolor{quarto-callout-important-color}{HTML}{CC1914}
\definecolor{quarto-callout-warning-color}{HTML}{EB9113}
\definecolor{quarto-callout-tip-color}{HTML}{00A047}
\definecolor{quarto-callout-caution-color}{HTML}{FC5300}
\definecolor{quarto-callout-color-frame}{HTML}{acacac}
\definecolor{quarto-callout-note-color-frame}{HTML}{4582ec}
\definecolor{quarto-callout-important-color-frame}{HTML}{d9534f}
\definecolor{quarto-callout-warning-color-frame}{HTML}{f0ad4e}
\definecolor{quarto-callout-tip-color-frame}{HTML}{02b875}
\definecolor{quarto-callout-caution-color-frame}{HTML}{fd7e14}
\makeatother
\makeatletter
\@ifpackageloaded{bookmark}{}{\usepackage{bookmark}}
\makeatother
\makeatletter
\@ifpackageloaded{caption}{}{\usepackage{caption}}
\AtBeginDocument{%
\ifdefined\contentsname
  \renewcommand*\contentsname{Table of contents}
\else
  \newcommand\contentsname{Table of contents}
\fi
\ifdefined\listfigurename
  \renewcommand*\listfigurename{List of Figures}
\else
  \newcommand\listfigurename{List of Figures}
\fi
\ifdefined\listtablename
  \renewcommand*\listtablename{List of Tables}
\else
  \newcommand\listtablename{List of Tables}
\fi
\ifdefined\figurename
  \renewcommand*\figurename{Figure}
\else
  \newcommand\figurename{Figure}
\fi
\ifdefined\tablename
  \renewcommand*\tablename{Table}
\else
  \newcommand\tablename{Table}
\fi
}
\@ifpackageloaded{float}{}{\usepackage{float}}
\floatstyle{ruled}
\@ifundefined{c@chapter}{\newfloat{codelisting}{h}{lop}}{\newfloat{codelisting}{h}{lop}[chapter]}
\floatname{codelisting}{Listing}
\newcommand*\listoflistings{\listof{codelisting}{List of Listings}}
\makeatother
\makeatletter
\makeatother
\makeatletter
\@ifpackageloaded{caption}{}{\usepackage{caption}}
\@ifpackageloaded{subcaption}{}{\usepackage{subcaption}}
\makeatother

\usepackage{bookmark}

\IfFileExists{xurl.sty}{\usepackage{xurl}}{} % add URL line breaks if available
\urlstyle{same} % disable monospaced font for URLs
\hypersetup{
  pdftitle={CJILS Copyediting and Layout Manual},
  pdfauthor={Philippe Mongeon},
  colorlinks=true,
  linkcolor={blue},
  filecolor={Maroon},
  citecolor={Blue},
  urlcolor={Blue},
  pdfcreator={LaTeX via pandoc}}


\title{CJILS Copyediting and Layout Manual}
\author{Philippe Mongeon}
\date{2025-07-19}

\begin{document}
\maketitle

\renewcommand*\contentsname{Table of contents}
{
\hypersetup{linkcolor=}
\setcounter{tocdepth}{2}
\tableofcontents
}

\bookmarksetup{startatroot}

\chapter*{Introduction}\label{introduction}
\addcontentsline{toc}{chapter}{Introduction}

\markboth{Introduction}{Introduction}

Welcome to the official guide for supporting the copyediting process of
manuscripts submitted to the \emph{Canadian Journal of Information and
Library Science (CJILS)}. This resource is designed to assist the
journal editors, managers, and copyeditors editors in preparing CJILS
articles for publication---ensuring clarity, consistency, and compliance
with journal style.

Whether you're a seasoned language professional or newly joining the
CJILS editorial workflow, this guide provides step-by-step instructions,
templates, checklists, and formatting support tailored to the specific
needs of our journal.

\section*{Contents}\label{contents}
\addcontentsline{toc}{section}{Contents}

\markright{Contents}

\subsection*{Section 1. Copyediting Process and
Tools}\label{section-1.-copyediting-process-and-tools}
\addcontentsline{toc}{subsection}{Section 1. Copyediting Process and
Tools}

\begin{itemize}
\tightlist
\item
  \href{general.qmd}{Overview}
\item
  \href{ojs.qmd}{Open Journal System (OJS)}
\item
  \href{overleaf.qmd}{Overleaf}
\item
  \href{checklist.qmd}{Checklist}
\end{itemize}

\subsection*{Section 2. LaTeX guide}\label{section-2.-latex-guide}
\addcontentsline{toc}{subsection}{Section 2. LaTeX guide}

\begin{itemize}
\tightlist
\item
  \href{header.qmd}{Header}
\item
  \href{title-page.qmd}{Title Page}
\item
  \href{main.qmd}{Main Text}
\item
  \href{figures.qmd}{Figures}
\item
  \href{tables.qmd}{Tables}
\item
  \href{references.qmd}{References}
\item
  \href{appendices.qmd}{Appendices}
\end{itemize}

\part{Copyediting Process and Tools}

\chapter*{Overview}\label{overview}
\addcontentsline{toc}{chapter}{Overview}

\markboth{Overview}{Overview}

\section*{Introduction}\label{introduction-1}
\addcontentsline{toc}{section}{Introduction}

\markright{Introduction}

\section*{Manuscript submission}\label{manuscript-submission}
\addcontentsline{toc}{section}{Manuscript submission}

\markright{Manuscript submission}

\section*{Peer-review}\label{peer-review}
\addcontentsline{toc}{section}{Peer-review}

\markright{Peer-review}

\section*{Copyediting}\label{copyediting}
\addcontentsline{toc}{section}{Copyediting}

\markright{Copyediting}

\subsection*{Accepting an invitation to copyedit a
manuscript}\label{accepting-an-invitation-to-copyedit-a-manuscript}
\addcontentsline{toc}{subsection}{Accepting an invitation to copyedit a
manuscript}

\subsection*{Accessing the}\label{accessing-the}
\addcontentsline{toc}{subsection}{Accessing the}

\subsection*{\#}\label{section}
\addcontentsline{toc}{subsection}{\#}

\section*{Production}\label{production}
\addcontentsline{toc}{section}{Production}

\markright{Production}

\section*{Publication}\label{publication}
\addcontentsline{toc}{section}{Publication}

\markright{Publication}

\chapter*{Open Journal System}\label{open-journal-system}
\addcontentsline{toc}{chapter}{Open Journal System}

\markboth{Open Journal System}{Open Journal System}

This is a Quarto book.

To learn more about Quarto books visit
\url{https://quarto.org/docs/books}.

\begin{Shaded}
\begin{Highlighting}[]
\DecValTok{1} \SpecialCharTok{+} \DecValTok{1}
\end{Highlighting}
\end{Shaded}

\begin{verbatim}
[1] 2
\end{verbatim}

\chapter*{Overleaf}\label{overleaf}
\addcontentsline{toc}{chapter}{Overleaf}

\markboth{Overleaf}{Overleaf}

Upcoming\ldots.

\chapter*{Checklist}\label{checklist}
\addcontentsline{toc}{chapter}{Checklist}

\markboth{Checklist}{Checklist}

Upcoming\ldots{}

\part{LaTeX}

\chapter*{Header}\label{header}
\addcontentsline{toc}{chapter}{Header}

\markboth{Header}{Header}

\section*{First page header}\label{first-page-header}
\addcontentsline{toc}{section}{First page header}

\markright{First page header}

The first page header contains the following information

\begin{itemize}
\tightlist
\item
  The journal title in both language and capitalized (should already be
  in the template).
\item
  The volume number
\item
  The issue number
\item
  DOI
\item
  Document type
\end{itemize}

\begin{Shaded}
\begin{Highlighting}[]
\NormalTok{\textbackslash{}journal\{\textbackslash{}scriptsize\{The Canadian Journal of Information and Library Science {-} La Revue canadienne des sciences de l\textquotesingle{}information et de bibliothéconomie (CJILS{-}RCSIB)\}\}}
\NormalTok{\textbackslash{}volume\{\textbackslash{}scriptsize\{Vol. vv, No. i (yyyy) }
\NormalTok{\textbackslash{}\textbackslash{} }
\NormalTok{DOI: 10.5206/cjils{-}rcsib.v{-}{-}i{-}.{-}{-}{-}{-}{-}}
\NormalTok{\textbackslash{}\textbackslash{} }
\NormalTok{\textbackslash{}textbf\{\textbackslash{}textit\{Conference paper\}\}\}\} \% Document type }
\end{Highlighting}
\end{Shaded}

test \#\#\# Document type

\section*{Running Head}\label{running-head}
\addcontentsline{toc}{section}{Running Head}

\markright{Running Head}

\subsection*{Odd Pages}\label{odd-pages}
\addcontentsline{toc}{subsection}{Odd Pages}

gfsdfgd

\begin{Shaded}
\begin{Highlighting}[]
\NormalTok{\textbackslash{}shorttitle\{CJILS/RCSIB Vol. 47, No. 2 (2024). DOI: 10.5206/cjils{-}rcsib.v{-}i{-}{-}.{-}{-}\}}
\end{Highlighting}
\end{Shaded}

\subsection*{Even Pages}\label{even-pages}
\addcontentsline{toc}{subsection}{Even Pages}

gdfg

\begin{Shaded}
\begin{Highlighting}[]
\NormalTok{\textbackslash{}leftheader\{Lastname and Lastname\}}
\end{Highlighting}
\end{Shaded}

test

\section*{Summary}\label{summary}
\addcontentsline{toc}{section}{Summary}

\markright{Summary}

\chapter*{Title page}\label{title-page}
\addcontentsline{toc}{chapter}{Title page}

\markboth{Title page}{Title page}

\section*{Title}\label{title}
\addcontentsline{toc}{section}{Title}

\markright{Title}

\section*{Abstract}\label{abstract}
\addcontentsline{toc}{section}{Abstract}

\markright{Abstract}

\section*{Keywords}\label{keywords}
\addcontentsline{toc}{section}{Keywords}

\markright{Keywords}

\section*{Authors}\label{authors}
\addcontentsline{toc}{section}{Authors}

\markright{Authors}

\section*{ORCID}\label{orcid}
\addcontentsline{toc}{section}{ORCID}

\markright{ORCID}

\section*{Affiliations}\label{affiliations}
\addcontentsline{toc}{section}{Affiliations}

\markright{Affiliations}

\section*{Authors with Multiple
Affiliations}\label{authors-with-multiple-affiliations}
\addcontentsline{toc}{section}{Authors with Multiple Affiliations}

\markright{Authors with Multiple Affiliations}

\section*{Corresponding Author}\label{corresponding-author}
\addcontentsline{toc}{section}{Corresponding Author}

\markright{Corresponding Author}

\section*{}\label{section-1}
\addcontentsline{toc}{section}{}

\markright{}

\chapter*{Main text}\label{main-text}
\addcontentsline{toc}{chapter}{Main text}

\markboth{Main text}{Main text}

\section*{Overview}\label{overview-1}
\addcontentsline{toc}{section}{Overview}

\markright{Overview}

\section*{Sections}\label{sections}
\addcontentsline{toc}{section}{Sections}

\markright{Sections}

\section*{Paragraphs}\label{paragraphs}
\addcontentsline{toc}{section}{Paragraphs}

\markright{Paragraphs}

\section*{Lists}\label{lists}
\addcontentsline{toc}{section}{Lists}

\markright{Lists}

\subsection*{Bullet points}\label{bullet-points}
\addcontentsline{toc}{subsection}{Bullet points}

\subsection*{Numbered lists}\label{numbered-lists}
\addcontentsline{toc}{subsection}{Numbered lists}

\subsection*{Spacing between list
items}\label{spacing-between-list-items}
\addcontentsline{toc}{subsection}{Spacing between list items}

\section*{Footnotes}\label{footnotes}
\addcontentsline{toc}{section}{Footnotes}

\markright{Footnotes}

\section*{Citations}\label{citations}
\addcontentsline{toc}{section}{Citations}

\markright{Citations}

\section*{Quotes}\label{quotes}
\addcontentsline{toc}{section}{Quotes}

\markright{Quotes}

\subsection*{Short quotes}\label{short-quotes}
\addcontentsline{toc}{subsection}{Short quotes}

\subsection*{Long quotes}\label{long-quotes}
\addcontentsline{toc}{subsection}{Long quotes}

\section*{Hyperlinks}\label{hyperlinks}
\addcontentsline{toc}{section}{Hyperlinks}

\markright{Hyperlinks}

\section*{Spelling}\label{spelling}
\addcontentsline{toc}{section}{Spelling}

\markright{Spelling}

\chapter*{Figures}\label{figures}
\addcontentsline{toc}{chapter}{Figures}

\markboth{Figures}{Figures}

Upcoming\ldots{}

\chapter*{Tables}\label{tables}
\addcontentsline{toc}{chapter}{Tables}

\markboth{Tables}{Tables}

\section*{One vs two column tables}\label{one-vs-two-column-tables}
\addcontentsline{toc}{section}{One vs two column tables}

\markright{One vs two column tables}

\subsection*{Single Column Table}\label{single-column-table}
\addcontentsline{toc}{subsection}{Single Column Table}

\begin{Shaded}
\begin{Highlighting}[]
\NormalTok{\textbackslash{}begin\{table\}}
\NormalTok{  \textbackslash{}begin\{tabular\}}
\NormalTok{  \textbackslash{}end\{tabular\}}
\NormalTok{\textbackslash{}end\{table\}}
\end{Highlighting}
\end{Shaded}

\subsection*{Two-Column Table}\label{two-column-table}
\addcontentsline{toc}{subsection}{Two-Column Table}

\begin{Shaded}
\begin{Highlighting}[]
\NormalTok{\textbackslash{}begin\{table*\}}
\NormalTok{  \textbackslash{}begin\{tabularx\}}
\NormalTok{  \textbackslash{}end\{tabularx\}}
\NormalTok{\textbackslash{}end\{table*\}}
\end{Highlighting}
\end{Shaded}

\subsection*{Long Tables}\label{long-tables}
\addcontentsline{toc}{subsection}{Long Tables}

\subsubsection*{single column}\label{single-column}
\addcontentsline{toc}{subsubsection}{single column}

\begin{verbatim}
\end{verbatim}

\subsubsection*{Two column}\label{two-column}
\addcontentsline{toc}{subsubsection}{Two column}

\begin{Shaded}
\begin{Highlighting}[]
\NormalTok{\textbackslash{}begin\{table*\}}
\NormalTok{  \textbackslash{}begin\{longtable\}}
\NormalTok{  \textbackslash{}end\{longtable\}}
\NormalTok{\textbackslash{}end\{table*\}}
\end{Highlighting}
\end{Shaded}

\section*{Table Position}\label{table-position}
\addcontentsline{toc}{section}{Table Position}

\markright{Table Position}

LaTeX will not always put your table exactly where you placed it in the
text due to its internal layout rules. \textbf{If} the default position
is not ideal, you can attempt giving LaTeX some instructions to better
position the table.

If you're happy with the table placement, you don't have to do anything.

\subsection*{Specifiers}\label{specifiers}
\addcontentsline{toc}{subsection}{Specifiers}

\begin{longtable}[]{@{}
  >{\raggedright\arraybackslash}p{(\linewidth - 2\tabcolsep) * \real{0.1528}}
  >{\raggedright\arraybackslash}p{(\linewidth - 2\tabcolsep) * \real{0.8472}}@{}}
\toprule\noalign{}
\begin{minipage}[b]{\linewidth}\raggedright
Specifier
\end{minipage} & \begin{minipage}[b]{\linewidth}\raggedright
Meaning
\end{minipage} \\
\midrule\noalign{}
\endhead
\bottomrule\noalign{}
\endlastfoot
h & Place the table \textbf{here}, roughly at the current location \\
t & Place the table at the \textbf{top} of the page \\
b & Place the table at the \textbf{bottom} of the page \\
p & Place the table on a \textbf{separate page} for floats only \\
! & Override LaTeX's internal float placement rules \\
H & Place the table \textbf{exactly here} (requires \texttt{float}
package) \\
\end{longtable}

\subsection*{Example}\label{example}
\addcontentsline{toc}{subsection}{Example}

\begin{Shaded}
\begin{Highlighting}[]
\NormalTok{\textbackslash{}begin\{table*\}}
\NormalTok{  \textbackslash{}begin\{longtable\}}
\NormalTok{  \textbackslash{}end\{longtable\}}
\NormalTok{\textbackslash{}end\{table*\}}
\end{Highlighting}
\end{Shaded}

\subsection*{Tips and tricks}\label{tips-and-tricks}
\addcontentsline{toc}{subsection}{Tips and tricks}

\begin{itemize}
\item
  You can combine them (e.g.,\texttt{{[}htbp!{]}} )to give LaTeX
  flexibility while nudging it toward your preferred location.
\item
  It is recommended not to use {[}h{]} alone, as it can sometimes be
  ignored. Prefer {[}ht{]} or {[}htbp{]}.
\end{itemize}

\begin{tcolorbox}[enhanced jigsaw, opacityback=0, toprule=.15mm, titlerule=0mm, colbacktitle=quarto-callout-tip-color!10!white, bottomrule=.15mm, colframe=quarto-callout-tip-color-frame, breakable, opacitybacktitle=0.6, coltitle=black, toptitle=1mm, left=2mm, bottomtitle=1mm, title=\textcolor{quarto-callout-tip-color}{\faLightbulb}\hspace{0.5em}{Do not fight LaTeX}, arc=.35mm, rightrule=.15mm, leftrule=.75mm, colback=white]

Often, you might find that latex is not at all cooperating with you when
it comes to positioning tables. My advice is to not die on that hill.
Your table is showing up somewhere in the document. It looks good, but
it's not where you want it, so you spend a few minutes trying different
strategies to move it to a better spot. No luck? Just move on.

\end{tcolorbox}

\section*{Caption}\label{caption}
\addcontentsline{toc}{section}{Caption}

\markright{Caption}

The caption is the first element in the table.

\begin{Shaded}
\begin{Highlighting}[]
\NormalTok{\textbackslash{}caption\{\textbackslash{}label\{tbl{-}mytable\}\textbackslash{}textit\{title of the table goes here\}\}}
\end{Highlighting}
\end{Shaded}

\section*{Columns}\label{columns}
\addcontentsline{toc}{section}{Columns}

\markright{Columns}

\begin{Shaded}
\begin{Highlighting}[]
\NormalTok{\textbackslash{}begin\{tabularx\}\{\textbackslash{}textwidth\}\{\%}
\NormalTok{  \textgreater{}\{\textbackslash{}raggedright\textbackslash{}arraybackslash\}p\{1.5in\} }
\NormalTok{  \textgreater{}\{\textbackslash{}raggedright\textbackslash{}arraybackslash\}p\{2.5in\}}
\NormalTok{  \textgreater{}\{\textbackslash{}raggedleft\textbackslash{}arraybackslash\}p\{2.5in\} }
\NormalTok{\}}
\end{Highlighting}
\end{Shaded}

\subsection*{Alignment}\label{alignment}
\addcontentsline{toc}{subsection}{Alignment}

\subsection*{Column width}\label{column-width}
\addcontentsline{toc}{subsection}{Column width}

\textbf{\{\textbackslash textwidth\}}

\textbf{p\{1.5in\}}

\section*{Rows}\label{rows}
\addcontentsline{toc}{section}{Rows}

\markright{Rows}

\section*{Borders}\label{borders}
\addcontentsline{toc}{section}{Borders}

\markright{Borders}

\section*{Notes}\label{notes}
\addcontentsline{toc}{section}{Notes}

\markright{Notes}

\section*{Layout}\label{layout}
\addcontentsline{toc}{section}{Layout}

\markright{Layout}

\section*{Tips and tricks}\label{tips-and-tricks-1}
\addcontentsline{toc}{section}{Tips and tricks}

\markright{Tips and tricks}

\subsection*{Copy pasting from Word}\label{copy-pasting-from-word}
\addcontentsline{toc}{subsection}{Copy pasting from Word}

\subsection*{Copilot}\label{copilot}
\addcontentsline{toc}{subsection}{Copilot}

\chapter*{References}\label{references}
\addcontentsline{toc}{chapter}{References}

\markboth{References}{References}

upcoming\ldots.

\phantomsection\label{refs}
\begin{CSLReferences}{0}{1}
\end{CSLReferences}

\chapter*{Appendices}\label{appendices}
\addcontentsline{toc}{chapter}{Appendices}

\markboth{Appendices}{Appendices}

Upcoming\ldots{}




\end{document}
